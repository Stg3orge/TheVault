\section{Introduktion}\label{intro}
Denne rapport beskriver de overvejelser og beslutninger der er taget i forhold til planlægning, bestemmelser, metodevalg, kvalitetssikring, arkitektur og en reflektion på førnævnte. Rapporten vil først beskrive den vision gruppen havde for systemet og herefter agile arbejdsmetoder, da disse er grundlaget for hvordan arbejdet er udført igennem projektperioden i forhold til denne vision.

\nl

Efter disse beskrivelser vil der blive sammenlignet med plandrevne arbejdsmetoder da disse var benyttet i forrige semester og er en ældre arbejdsform. Herefter vil der blive beskrevet hvad der blev valgt som arbejdsmetode og hvorfor. Herunder vil de sprints der er gennemført blive analyseret og reflekteret over også. Så vil der blive opstillet nogle kvalitetskriterier og reflekteret over om disse blev opfyldt for det endelige produkt. Hen af slutningen af rapporten vil der blive reflekteret og perspektiveret over alt førhenlæggende og konkluderet.

\section{Reflektion på agile arbejdsmetoder}
I dette afsnit vil der blive kigget på hvordan de agile arbejdsmetoder har fungeret i projektet. Der vil blive kigget på begge i forhold til projektet og til sidst reflekteret over hvad gruppen fik ud af hver.
\subsection{SCRUM}
Scrum er en agil metode til at styre et projekt på. Agilt softwareudvikling med SCRUM kan oftes ses som en normal metode, men det er mere et framework for at udføre forskellige former af arbejde på, der er langt mere frihed i SCRUM end i XP.

\nl

I stedet for at give detaljerede beskrivelser af, hvordan alt skal gøres i et projekt, er meget af det lagt op til udviklingsholdet. Dette skyldes, at holdet ved bedst, hvordan man løser det problem, de bliver præsenteret overfor.

\nl

I dette projekt blev der startet og kørt SCRUM fra det andet sprint af til slut. 
\subsection{XP}
I eXtreme Programming er der lagt regler op for hvordan arbejdsprocessen forløber. Dette er i stor modsætning til SCRUM der ikke sætter direkte disse krav til udviklingsholdet. XP er baseret på 4 værdier og 12 principper til at guide holdet til slutningen af en udviklingsperiode. I dette projekt blev der kørt XP i det første sprint hvorefter der blev kørt SCRUM i alle sprint siden.

\nl

Af de 12 principper var principperne simpelt design, test først, kontinuert integration og kode standarder de mest prevalente i det ene sprint hvor XP blev kørt. Der blev også forsøgt planlægning med planning poker, men der var nogle misforståelser og uheld med dette så det blev (lagt på hylden?) kort tid inde i det første sprint.

\nl


\section{Planlægning og Kvalitetssikkerhed}\label{planl}
I dette afnsit vil der blive gennemgået de overvejelser der er taget på at sikre kvaliteten af projektets produkt og den planlægning der er indgået i formuleringen og udarbejdelsen af det. For at have et udgangspunkt vil der blive refereret til system visionen fra afsnit \ref{systemvision}.

\nl

\subsection{Test}
For at sikre kvalitet er der blevet kørt forskellige typer tests på produktet under udvikling. 