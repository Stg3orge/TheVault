\section{Visual Basic}

I dette afsnit vil der blive analyseret hvad Visual Basic er i relation til dette produkt, hvorfor den bliver analyseret og de inspirationer sproget giver til produktet i dette projekt.
Visual Basic er et grafisk objekt-orienteret programmeringssprog med tekst-baserede funktionaliteter. Det kan sammenlignes med WinForms bibliotektet i .NET \bf{frameworket}. Programmøren har mulighed for at bearbejde en konfigurerbar GUI sammentidigt med at dirigere egenskaber for objekterne i denne GUI. Disse moduler er principielt det samme mål at opnå i dette projekt med et fokus på en ung aldersgruppe som gennemgået i afsnit \red{AFSNIT}

\subsection{Analyse af Visual Basic}

Der kigges her generelt på Visual Basics muligheder for at tillægge kode til objekter der kan transformeres i en GUI. I dette projekts produkt vil der være fokus på brugerens muligheder for a programmere disse egenskaber selv, ikke automatisk generere dem, og derefter vise deres program visuelt i en GUI. Det anses derfor at udfordringen vil ligge i at lære brugeren at lægge grundlaget mellem visuelle elementer og at programmere elementerne fra bunden.

(Afsnit om læredygtigheden om børn her)
Børn(alder?) har lettere ved at lære og huske på logiske systemer med benyttelse af visuelle objektorienterede elementer end med ren tekstbaseret kode\cite{techagekids}. På baggrund af dette ligger der et godt grundlag for at inkorporere noget lignende til det Visual Basic tilbyder i et programmérbart objektorienteret system.

\input{Pictures/Problemanalyse/vb1.PNG}
På figur \ref{fig:vb1} kan man se et udsnit af de egenskaber et objekt kunne have i Visual Basics brugergrænseflade. Disse egenskaber kunne forestilles at blive programmeret fra bunden af programmøren i dette projekts endelige system.

http://www.pcadvisor.co.uk/how-to/software/how-learn-code-in-2016-microbit-3599820/
http://www.techagekids.com/2016/07/graphical-vs-text-based-coding-for-kids.html
https://codecombat.com/