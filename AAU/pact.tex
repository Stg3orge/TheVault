People
\textit{Målgruppen} er kursister, ansatte og den administrative gren af en faglig institution, netop VUC (Voksenuddannelsescenter). Størstedelen af kursisterne på institutionen er i aldersgruppen (18-??, need affirmation), hvilket betyder at målgruppen har en generel forståelse for benyttelse af webapplikationer. Dette projekts produkt er derimod specifikt konstrueret efter ønsker fra gruppens informant, der benytter et lignende system.
Som givet i Informant interviewet ville informanten gerne se et pop-up system til at minde om givne afleveringer og lignende opgaver. Dette er en motiverende faktor for informanten(?). Målgruppen er orienteret omkring danske brugere og er derfor primært skrevet på dansk tale.

Activities
Applikationen bliver primært designet til at give platform til kommunikation mellem kursister og ansatte. De løbende opdateringer der kan forekomme skal være lette for brugeren at finde og lette at blive orienteret omkring.
Brugeren har sin individuelle brugerprofil, med hans/hendes egne skemaer og meddelsesindbakker, samt adgang til information omkring alt hvad der forekommer deres studie. Da sysemet er en webapplikation kan brugeren kun udføre én opgave af gangen, såsom at aflevere et dokument eller kigge på skemaet. Webapplikationen skal være intuitivt nok, som en hjemmeside, til at kunne tage sig af eventuelle fejlbeskeder.

Context
Webapplikationen er browser-orienteret og er derfor et mobilt system. Websystemet kan derfor både blive benyttet på mobile og stationære systemer. Systemet vil ofte blive opdateret med nye arrangementer, møder, omlagte timer og skemaændringer mm.. Dette korrelere derfor også med at systemet bliver benyttet primært inden for skolesæson. Den administrative del vil dog opdatere det mest uden for sæson(?). Systemet skal kunne håndtere et større antal at brugere på én gang, f.eks. når karakterresultater bliver offentliggjort kan der befinde sig stress på systemet.

Det skal være muligt at brugeren skal kunne kommunikere gennem et mail-system til at arrangere arrangementer der kan stå til gavn for kompetenceudvikling i relation til de aktiviteter den studerende har interesse for.

Technologies
Webapplikationen kan blive interageret med via. fysisk kontakt i form af fingre eller perifer enhed såsom tastatur og mus. Da applikationen er browserbaseret vil de normale regler for interaktionsdesign på webbaserede systemer forbeholde. Dette inkluderer håndtering af frem- og tilbage funktionaliteter, seperate vinduer osv. 
Kommunikation foregår som sagt gennem et mail-system og løbende opdateringer til skema, arrangementer, møder mm..
Systemet kan inkluderer notifikationslyde til de ønskede pop-ups som informanten har nævnt.
Systemet skal hele tiden være tilgængeligt og online, da vigtig information bliver givet igennem det.
